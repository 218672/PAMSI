\documentclass[10pt, a4paper]{article}
\usepackage{geometry} 
\newgeometry{tmargin=1cm, bmargin=1cm, lmargin=1cm, rmargin=1cm} 
\usepackage[utf8]{inputenc}
\usepackage{polski}
\usepackage{enumerate}
\title{\textbf{PAMSI - Sprawozdanie 1}}
\author{\textbf{Filip Guzy 218672}}

\begin{document}
\maketitle

\begin{flushleft}
\textbf{Analiza} \newline \newline
W celu sprawdzenia efektywności różnych metod powiększania tablic dynamicznych podczas ich przepełniania zaimplementowano trzy różne algorytmy wykonujące tę czynność. Zastosowano do nich odpowiednie oznaczenia (gdzie k to rozmiar tablicy): 
\begin{enumerate}
\item $k=k+1$ - algorytm powiększający rozmiar tablicy dynamicznej o jeden w przypadku każdego przepełnienia, zapisany w pliku o nazwie tablice\_alg\_1.cpp. 
\item $k=2*k$ - algorytm podwajający rozmiar tablicy dynamicznej w przypadku każdego przepełnienia, zapisany w pliku o nazwie tablice\_alg\_2.cpp.
\item $k=k+100$ - algorytm powiększający rozmiar tablicy dynamicznej o 100 w przypadku każdego przepełnienia, zapisany w pliku o nazwie tablice\_alg\_3.cpp.
\end{enumerate}

Następnie testowano ich efektywność dla różnych ilości danych, odpowiednio: 10, 100, 1000, 1000000, 1000000000. Po uzyskaniu wynikach postanowiono sprawdzić ich efektywność również dla liczb 10000 oraz 100000, w celu lepszego ukazania rozbieżności między nimi. Na wejście programów podano ilość danych do wczytania, na wyjściu otrzymano czas (w milisekundach) zapełnienia tablicy odpowiednią ilością zer. Wyniki przedstawiono w poniższej tabeli:

\begin{table}[h]
\centering
\caption{}
\begin{tabular}{|c|c|c|c|c|c|c|c|} \hline
Nazwa algorytmu & 10 & $10^2$ & $10^3$ & $10^4$ & $10^5$ & $10^6$ & $10^9$ \\ \hline
$k=k+1$ & 0.00199997 & 0.0569999 & 3.737 & 283.313 & 28722.3 & - & - \\ \hline
$k=2*k$ & 0.00100005 & 0.0079999 & 0.027 & 0.173 & 2.374 & 34.052 & 26231.0 \\ \hline
$k=k+100$ & 0.00100005 & 0.00700009 & 0.0549999 & 3.010 & 333.233 & 32496.9 & - \\ \hline
\end{tabular}
\end{table}

Jak można zauważyć, z wpisaniem miliarda liczb do tablicy w rozsądnym czasie poradził sobie jedynie algorytm $n=2*n$. Jest to zatem najefektywniejszy z trzech algorytmów. Gorsze wyniki uzyskał algorytm $n=n+100$, który czas alokacji miliona liczb w tablicy miał porównywalny z czasem alokacji miliarda liczb przez algorytm $k=2*k$. Najmniej efektywnie prezentuje się algorytm $k=k+1$, którego czas alokacji stu tysięcy liczb był porównywalny z czasem alokacji miliona liczb przez algorytm $k=k+100$ i miliarda przez algorytm $k=2*n$. Różnice pomiędzy efektywnością zapełniania tablicy i realokacji jej pamięci zaczynają być widoczne już na poziomie tysiąca wczytanych liczb, gdzie różnica w czasie pomiędzy $k=k+1$ a dwoma pozostałymi jest na poziomie dwóch rzędów. Różnica pomiędzy $k=k+100$ a $k=2*k$ uwidoczniła się podczas wczytania dziesięciu tysięcy liczb, gdzie czas wykonania $k=2*k$ był mniejszy o jeden rząd wielkości. Różnice te są efektem częstości realokacji pamięci dynamicznej dla tablicy podczas jej przepełniania się. W przypadku $k=k+1$ po osiągnięciu pierwszego przepełnienia pamięć realokowana jest za każdym powtórzeniem operacji wpisania elementu. Kolejny, efektywniejszy $k=k+100$ ma stały odstęp wykonywania realokacji po pierwszym przepełnieiu, który wynosi 100 wpisanych liczb, czyli realokacja całej pamięci w tablicy jest wykonywana 100 razy rzadziej niż w algorytmie $k=k+1$. Najefektywniejszy z nich, $k=2*n$, z czasem wykonuje coraz mniej realokacji, co przy wzroście ilości danych jedynie działa na jego korzyść. \newline

\textbf{Złożoność obliczeniowa} \newline

Średni czas operacji inkrementalnego powiększania rozmiarów tablic w większości przypadków zawiera się w O($n^2$) w przypadku powiększania o const=1 oraz w O($n$) przy powiększaniu o większe stałe, natomiast w przypadku algorytmów podwajających średni czas operacji zawiera się w O($n$). \newline


\textbf{Wnioski} \newline

Z powyższej analizy wynika, że algorytmy podwajające rozmiar tablicy są efektywniejsze niż algorytmy inkrementalne.
\end{flushleft}
\end{document}

