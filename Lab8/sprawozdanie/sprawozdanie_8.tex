\documentclass[10pt, a4paper]{article}
\date{\today}
\usepackage{geometry}
\newgeometry{tmargin=1cm, bmargin=1cm, lmargin=1cm, rmargin=1cm}
\usepackage[utf8]{inputenc}
\usepackage{polski}
\usepackage{graphicx}
\usepackage{enumerate}
\title{\textbf{PAMSI - Sprawozdanie 8}}
\author{\textbf{Filip Guzy 218672}}

\begin{document}
\maketitle

\begin{flushleft}
\textbf{Graf nieskierowany ważony i algorytmy branch and bound} \newline

Graf nieskierowany ważony jest strukturą danych zbudowaną z wierzchołków i krawędzi, których wartości opisują relacje pomiędzy poszczególnymi wierzchołkami. W celu przetestowania algorytmów wyszukiwania najbardziej optymalnych tras pomiędzy wierzchołkami z rodziny branch and bound (tł. metoda podziału i ograniczeń) zaimplementowano je jako metody wyżej wspomnianego grafu. W celu ułatwienia realizacji obydwu przeszukań uzupełniono całą strukturę o macierz kosztów opisującą wagi poszczególnych krawędzi przy zachowaniu istniejącej już listy sąsiedztwa. Algorytm branch and bound został zaimplementowany w dwóch wersjach: bez listy rozszerzonych wierzchołków (extended list) oraz z listą. Realizacja obu przeszukań została zrealizowana na bazie zaimplementowanej w tym celu kolejki priorytetowej. \newline

\textbf{Pomiary czasów działania i rozszerzanych ścieżek} \newline

Algorytmy branch and bound charakteryzują się różną złożonością w zależności od kilku czynników. Przede wszystkim należy wziąć pod uwagę to, że wybrana implementacja jest oparta na algorytmie przeszukiwania BFS (Best-First Search). W tym przypadku złożoność czasowa działania algorytmu jest zależna od liczby sąsiadów dowolnego wierzchołka, do których koszt przejścia jest zbliżony do kosztu osiągnięcia tego tego wierzchołka lub jest równy zero. Biorąc pod uwagę fakt, że prawdopodobieństwa wylosowania wartości wag były równe (rozkład równomierny) oraz że każdy wierzchołek posiadał kilku sąsiadów można przyjąć, że większość wierzchołków miała sąsiada o zbliżonym koszcie przejścia. Zgodnie z takimi założeniami, złożoność czasowa testowanego przypadku powinna wynosić O($n^2$), gdzie n jest liczbą wierzchołków. Poniżej przedstawiono trzy różne możliwości złożoności czasowej tego algorytmu. Oznaczmy jako L liczbę sąsiadów pojedynczego wierzchołka o zbliżonym koszcie przejścia.

\begin{enumerate}
\item O($n$), gdy $L>1$
\item O($n^2$), gdy $L=1$
\item O($k^n$), gdy $L<1$ i $k$ jest pewną stałą wartością.
\end{enumerate}

W celu sprawdzenia przewidywanej złożoności przeprowadzono pomiary czasu oraz rozwiniętych ścieżek dla grafów o rozmiarach 10, 100, 1000, 10000. W poniższych tabelach przedstawiono pomiary dla wersji branch and bound bez listy rozszerzonych wierzchołków oraz z listą, a także wykresy porównujące obie wersje implementacji.


\begin{table}[!h]
\centering
\caption{Branch and bound - czas wyszukiwania}
\begin{tabular}{|c|c|c|c|c|c|c|} \hline 
Ilość danych & 10 & 100 & 1000 & 10000 \\ \hline
Czas [ms] & 0,049 & 0,155 & 19,074 & 2738,1 \\
& 0,0370001 & 0,176 & 19,601 & 2753,46 \\
& 0,0359999 & 0,16 & 21,368 & 2686,3 \\
& 0,036 & 0,152 & 22,87 & 2780,22 \\
& 0,041 & 0,152 & 24,204 & 2734,58 \\
& 0,041 & 0,162 & 21,393 & 2847,54 \\
& 0,039 & 0,16 & 22,603 & 2767,24 \\
& 0,046 & 0,156 & 22,526 & 2740,64 \\
& 0,0400001 & 0,161 & 23,377 & 2758,81 \\
& 0,041 & 0,166 & 23,52 & 2758,19  \\ \hline
Średnia & 0,04060001 & 0,16 & 22,0536 & 2756,508 \\ \hline
\end{tabular}
\end{table}

\begin{table}[!h]
\centering
\caption{Branch and bound - rozwinięte ścieżki}
\begin{tabular}{|c|c|c|c|c|c|c|} \hline 
Ilość danych & 10 & 100 & 1000 & 10000 \\ \hline
Rozwinięte ścieżki  & 18 & 239 & 2489 & 24993 \\
& 14 & 239 & 2478 & 24989 \\
& 14 & 247 & 2498 & 24989 \\
& 14 & 239 & 2493 & 24980 \\
& 14 & 239 & 2483 & 24989 \\
& 30 & 247 & 2493 & 24989 \\
& 14 & 247 & 2489 & 24994 \\
& 25 & 243 & 2493 & 24989 \\
& 18 & 250 & 2498 & 24993 \\
& 18 & 251 & 2489 & 24989 \\ \hline
Średnia & 17,9 & 244,1 & 2490,3 & 24989,4 \\ \hline
\end{tabular}
\end{table}

\begin{table}[!h]
\centering
\caption{Branch and bound with extended list - czas wyszukiwania}
\begin{tabular}{|c|c|c|c|c|c|c|} \hline 
Ilość danych & 10 & 100 & 1000 & 10000 \\ \hline
Czas [ms] & 0,00899994 & 0,096 & 4,034 & 761,906 \\
& 0,00800002 & 0,0749999 & 3,942 & 781,769 \\
& 0,00800002 & 0,0749999 & 4,09 & 770,823 \\
& 0,00800002 & 0,0749999 & 5,397 & 798,609 \\
& 0,00800002 & 0,0749999 & 5,01 & 781,514 \\
& 0,00800002 & 0,076 & 3,907 & 748,453 \\
& 0,00800002 & 0,0799999 & 5,611 & 768,92 \\
& 0,00900006 & 0,0779999 & 4,164 & 787,121 \\
& 0,00900006 & 0,077 & 4,785 & 776,154 \\
& 0,00899994 & 0,0779999 & 4,119 & 778,916 \\ \hline
Średnia & 0,008400012 & 0,07849993 & 4,5059 & 775,4185 \\ \hline
\end{tabular}
\end{table}

\begin{table}[!h]
\centering
\caption{Branch and bound with extended list - rozwinięte ścieżki}
\begin{tabular}{|c|c|c|c|c|c|c|} \hline 
Ilość danych & 10 & 100 & 1000 & 10000 \\ \hline
Rozwinięte ścieżki & 12 & 147 & 1497 & 14994 \\
& 11 & 147 & 1493 & 14997 \\
& 11 & 146 & 1497 & 14997 \\
& 11 & 147 & 1497 & 14994 \\
& 11 & 147 & 1494 & 14997 \\
& 15 & 146 & 1497 & 14997 \\
& 11 & 146 & 1497 & 14997 \\
& 14 & 147 & 1497 & 14997 \\
& 12 & 147 & 1496 & 14997 \\
& 12 & 147 & 1497 & 14997 \\ \hline
Średnia & 12 & 146,7 & 1496,2 & 14996,4 \\ \hline
\end{tabular}
\end{table}

\newpage

\begin{figure}[!h]
\centering
\includegraphics[width=13cm]b
\label{fig:obrazek b}
\end{figure}


\begin{figure}[!h]
\centering
\includegraphics[width=14cm]e
\label{fig:obrazek e}
\end{figure}

\newpage
Jak można zauważyć z rysunków, przeszukiwanie branch and bound wykonuje się w czasie kwadratowym, zatem złożoność jest zgodna z teorią. Ilość rozwijanych ścieżek rośnie liniowo. Po przeanalizowaniu obu wersji algorytmu można stwierdzić, że zastosowanie listy odwiedzonych wierzchołków umożliwiło znaczne polepszenie czasu wykonywania oraz zmniejszenie ilości rozwijanych ścieżek, przez co algorytm stał się skuteczniejszy w wyszukiwaniu optymalnej ścieżki. \newline

\textbf{Wnioski} \newline

Z przeprowadzonych pomiarów wynika, że algorytmy branch and bound w skuteczny sposób umożliwiają odnalezienie optymalnej ścieżki pomiędzy dwoma wierzchołkami. Zaimplementowane przeszukiwania mają złożoność obliczeniową zgodną z teorią. Zastosowanie dodatkowej listy zawierającej informacje o odwiedzonych wierzchołkach umożliwia znaczne polepszenie efektywności wyznaczania optymalnych tras. \newline \newline \newline

\textbf{Bibliografia} \newline \newline
Weixiong Zhang, \textit{Branch-and-bound Search Algorithms and Their Computational Complexity}, May 1996

\end{flushleft}

\end{document}