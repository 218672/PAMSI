\documentclass[10pt, a4paper]{article}
\usepackage{geometry}
\newgeometry{tmargin=1cm, bmargin=1cm, lmargin=1cm, rmargin=1cm}
\usepackage[utf8]{inputenc}
\usepackage{polski}
\usepackage{enumerate}
\usepackage{graphicx}
\title{\textbf{PAMSI - Sprawozdanie 3}}
\author{\textbf{Filip Guzy 218672}}

\begin{document}
\maketitle

\begin{flushleft}
\textbf{Implementacja struktur danych} \newline \newline
W celu realizacji struktur danych takich jak lista, stos i kolejka zaimplementowano trzy interfejsy: IList, IStack oraz IQueue. Zawierają one wirtualne metody opisujące każdą ze struktur. Na bazie interfejsu listy jest możliwe zaimplementowanie dwóch pozostałych typów danych, jednak aby zwiększyć przejrzystość projektowanych struktur lepiej zastosować dla nich oddzielne interfejsy. Jednym z istotnych ograniczeń dla języka C++ jest możliwość dziedziczenia metod tylko po jednym interfejsie, w przeciwieństwie do Javy lub C\#, gdzie programista ma możliwość tworzenia obiektu na bazie kilku interfejsów. Opisy metod oraz sposoby ich działania zostały zawarte w plikach nagłówkowych odpowiednich implementacji ADT. Warto również uwzględnić, że są to implementacje węzłowe, wykorzystujące dynamicznie alokowane węzły jako kolejne elementy listy, kolejki lub stosu, co będzie miało znaczenie przy wykonywanych poniżej pomiarach. \newline

\textbf{Przeszukiwanie listy} \newline \newline
W implementacji klasy listy zawarto metodę przeznaczoną do sprawdzania czasu przeszukiwania listy dla dowolnej ilości danych. W celu zbadania złożoności obliczeniowej przeszukiwania list o różnych rozmiarach zmierzono czasy (w milisekundach) wykonania tej czynności dla następujących danych wejściowych: 10, 100, 1000, 1000000, 10000000. Nie wykonano pomiarów dla 1000000000 elementów, ponieważ w implementacji listy wykorzystano węzły alokowane dynamicznie jako kolejne elementy listy, przez co dla danych wejściowych wymagana pamięć przekroczyła dostępną pamięć RAM maszyny wirtualnej. Pomiary dla każdej ilości danych wejściowych powtórzono dziesięciokrotnie, a wyniki przedstawiono w poniższej tabeli:

\begin{table}[h]
\centering
\begin{tabular}{|c|c|c|c|c|} \hline
10 & 100 & 1000 & 1000000 & 10000000 \\ \hline
0,00199997 & 0,00100005 & 0,00999999 & 6,898 & 70,193 \\
0,000999928 & 0,000999928 & 0,0140001 & 6,832 & 72,634 \\
0,00199997 & 0,000999928 & 0,015 & 6,998 & 87,602 \\
0,00100005 & 0,000999928 & 0,00999999 & 7,009 & 71,251 \\
0,00100005 & 0,000999928 & 0,0140001 & 7,077 & 69,218 \\
0,000999928 & 0,000999928 & 0,0129999 & 6,842 & 71,151 \\
0,00100005 & 0,00100005 & 0,0139999 & 6,978 & 71,436 \\
0,00100005 & 0,000999928 & 0,0120001 & 6,961 & 68,91 \\
0,000999928 & 0,000999928 & 0,013 & 7,224 & 71,44 \\
0,000999928 & 0,000999928 & 0,00800002 & 6,913 & 72,68 \\ \hline
\end{tabular}
\end{table}

Dla każdej ilości danych wejściowych wyznaczono średnią arytmetyczną. Wyniki przedstawiono w poniższej tabeli:

\begin{table}[h]
\centering
\begin{tabular}{|c|c|c|c|c|} \hline
10 & 100 & 1000 & 1000000 & 10000000 \\ \hline
0,0011999852 & 0,0009999524 & 0,01230001 & 6,9732 & 72,6515 \\ \hline
\end{tabular}
\end{table}

Otrzymane dane przedstawiono również na wykresie:
\begin{figure}
\centering
\includegraphics[width=12cm]w
\label{fig:obrazek w}
\end{figure} \newline

Aby sprawdzić, czy pomiary są zgodne z założeniami przeanalizowano fragment kodu odpowiedzialny za przeszukanie tablicy: \newline

while(tmp$->$next) $\{$ // przechodzimy wszystkie wezly \newline
if(tmp$->$elem$>$0) $\{$ \newline
// jeśli znajdziemy szukany element, robimy jakąś operację, tutaj żadną \newline
$\}$ \newline
else // jeżeli nie to \newline
tmp=tmp$->$next; // zmieniamy węzeł na kolejny \newline
$\}$ \newline \newline

Wyrażenia warunkowe if i else mają złożoność obliczeniową równą O(1), tak jak operacja zmiany węzła, która następuje po każdym wywołaniu instrukcji else, więc wnętrze pętli while ma złożoność obliczeniową równą O(1). Biorąc pod uwagę pętlę while, która została użyta do przejścia przez wszystkie węzły listy, których jest n, złożoność obliczeniowa całego algorytmu przeszukania powinna wynosić O(n). Tak też jest w rzeczywistości: każde dziesięciokrotne zwiększenie ilości danych wejściowych powoduje dziesięciokrotne zwiększenie czasu wykonania algorytmu, zatem rzeczywista złożoność operacji jest liniowa, czyli zawiera się w O(n). Dysponując możliwością zmierzenia czasu dla większych ilości danych ta zależność byłaby lepiej widoczna na wykresie, jednak biorąc pod uwagę komplikacje z systemem nie udało się tego pokazać.  \newline

\textbf{Wnioski} \newline \newline
Otrzymane pomiary wskazują na to, że lista została zaimplementowana poprawnie i wszystkie operacje są wykonywane w odpowiednim czasie. Złożoność obliczeniowa algorytmu przeszukiwania listy należy do O(n), zatem jest zgodna z teorią. \newline

Implementacja listy wykorzystująca węzły wykorzystuje więcej pamięci niż implementacja oparta np. na tablicy statycznej, jednak ograniczenie miejsca w przypadku tej drugiej uniemożliwia elastyczne operowanie miejscem w strukturze przy wykonywaniu praktycznych projektów. 

\end{flushleft}

\end{document}